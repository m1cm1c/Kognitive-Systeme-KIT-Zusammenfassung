% !TeX root = summary.tex

\section*{Einleitung}

\subsection*{Anmerkung zur Zusammenfassung für SS14 Klausur}
Diese Version beinhaltet kleine Änderungen und Ergänzungen speziell für die SS 14 Klausur. Ein Hauptaugenmerk war es Teile die nicht klausurrelevant sind zu entfernen. Diese Teile können aber für spätere Klausuren relevant sein! Außerdem habe ich die Kapitelstruktur so geändert, dass es der zeitlichen Abfolge der Vorlesung entspricht. Ich habe auch Teile hinzugefügt die in der Version von Adam Urban nicht drin waren (Kapitel: Visuelle Wahrnehmung des Menschen). Ich hoffe diese sind für andere Studenten hilfreich. In diesem Sinne würde ich mich freuen wenn weitere Personen diese Zusammenfassung als Basis für spätere Klausuren aufgreifen.


\subsection*{Anmerkung zur Zusammenfassung von Adam Urban}

Dies ist der Versuch einer Zusammenfassung der Vorlesung \textsc{Kognitive Systeme} aus dem Sommersemester 2006 an der Universität Karlsruhe (TH). Sie erhebt weder Anspruch auf Vollständigkeit, noch auf Korrektheit. Dieses Dokument hält sich sehr stark an den Folien von Prof. R. Dillmann und Prof. A. Waibel. \\
Bei Fehlern würde ich mich über eine eMail an \href{mailto:adam.urban@gmail.com}{adam.urban@gmail.com} freuen.

\section*{Links}

Meine Homepage: \hfill \href{http://adam.urban.de.vu/}{adam.urban.de.vu} \\[0,2cm]
Offizielle KogSys Seite: \hfill \href{http://wwwiaim.ira.uka.de/Teaching/VorlesungKogSys/}{http://wwwiaim.ira.uka.de/Teaching/VorlesungKogSys/}

\cleardoublepage

\pagestyle{empty}

\begin{flushright}
\textsl{für die Community}
\end{flushright}


\vspace{8cm}

\begin{center}
\textsl{"{}Gebildet ist, wer weiß, \\ wo er findet, was er nicht weiß."{} \\[0,5cm] \qquad \qquad \qquad \qquad G. Simmel}
\end{center}
