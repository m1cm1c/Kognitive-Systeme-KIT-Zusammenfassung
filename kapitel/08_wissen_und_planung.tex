% !TeX root = summary.tex

\chapter{Wissen und Planung}
\section{Wissen}
\subsection{Einführung}

\subsubsection*{Was ist Wissen?}
Gespeicherte
\begin{itemize}
\item Beschreibungen, Modelle, Vorschriften, Motorische, kognitive Fähigkeiten
\end{itemize}
konkreter:
\begin{itemize}
\item Computerprogramme, Regeln der Aussagenlogik, Gewichte von Neuronalen Netzen, \dots
\end{itemize}

\subsection{Grundlagen}

% Bild Folie 8
\mypic{6}{wb1}

\begin{description}
	\item[Wissensdatenbank\index{Wissensdatenbank}] Sätze (Annahmen), die in einer Wissensrepräsentationssprache abgefasst sind.
\end{description}

\begin{center}
\begin{tabular}{ll}
$\alpha$: & "{}\textbf{Ident1} IST-EIN Fahrrad."{} \\  $\gamma$: & "{}\textbf{Ident3} BESITZT Ident1."{} oder \\ $\alpha$: & "{}$r \in R^{20}$."{}
\end{tabular}
\end{center}

% Bild Folie 9
\mypic{8}{wb2}

\begin{description}
\item[Erkläre:] Füge der Datenbank Wissen hinzu.
\item[Befrage:] Frage Wissen aus Datenbank ab.
\end{description}

% Bild Folie 10
\mypic{8}{wb3}

\begin{description}
\item[Deduktion\index{Deduktion}:] Leite neue Sätze von bekannten ab. Injektive Abbildung Deduktion: $$DB \to DB$$ Minimale Deduktion: Identität.
\end{description}

\subsection{Logik allgemein}

\begin{description}
	\item[allg. Logik] Elementarwerte, Symbolmenge, Belegungsmenge (alle gültigen Belegungen), Belegung/Modell, Syntax (legt fest, was wohlgeformt ist), Semantik (legt fest, welche Wahrheitswerte bei gegebener Belegung zugewiesen werden)
\end{description}

\subsubsection*{Folgerung $\models$}

$$\alpha \models \beta \, : \, \textrm{"{}} \beta \textrm{ folgt aus } \alpha \textrm{ oder für alle Modelle, in denen } \alpha \textrm{ wahr ist, ist } \beta \textrm{ ebenfalls wahr}$$

\subsubsection*{Wissensbasis}
Kann als $\wedge$-verknüpfte Sequenz von Sätzen aufgefasst werden:
$$WB \models \beta \quad \textrm{ genau dann wenn } \quad \left( \bigwedge\limits_{\alpha_i \in WB} \alpha_i \right) \Rightarrow \beta = wahr$$

\subsubsection*{Modellprüfung}

Algorithmus zur Überprüfung einer $WB \models \alpha$-Relation:
\begin{enumerate}
\item Finde Menge aller Modelle $M = \{M_i\}$ über $S$
\item Eliminiere alle $M_i$ mit $WB/M = falsch$
\item $WB \models \alpha$ genau dann, wenn $$\forall M_i \in M \, : \, \alpha / M_i = wahr$$
\end{enumerate}

\subsubsection*{Deduktion}

Wenn ein Algorithmus $i$ existiert, der den Satz $\alpha$ aus der Wissensbasis $WB$ ableiten kann, schreiben wir $$WB \vdash_i \alpha$$
("{}$\alpha$ wird durch $i$ aus $WB$ abgeleitet"{} oder "{}$i$ leitet $\alpha$ aus $WB$ ab"{}) \\[0,1cm]
Eigenschaften von Deduktionsalgorithmen:
\begin{itemize}
\item Algorithmus $i$ "{}korrekt"{} genau dann, wenn er \underline{nur} Sätze aus $WB$ ableitet, die aus $WB$ folgen: $$\forall \alpha \, : \, WB \vdash_i \alpha \quad \Rightarrow \quad WB \models \alpha$$
\item Algorithmus $i$ "{}vollständig"{} genau dann, wenn er \underline{alle} Sätze aus $WB$ ableitet, die aus $WB$ folgen: $$\forall \alpha \, : \, WB \models \alpha \quad \Rightarrow \quad WB \vdash_i \alpha$$
\end{itemize}

\subsubsection*{Zusammenfassung}

\begin{description}
	\item[Logik] bestimmt Wahrheitsgehalt von Sätzen in Bezug auf Modelle
	\item[Wissensbasis] Menge von Sätzen das zu einem Modell passt oder nicht
\end{description}

\section{Aussagenlogik}

\subsection{Syntax}

\begin{eqnarray*}
Satz &\to& Atom \, | \, Komplex \\ Atom &\to& True \, | \, False \, | \, Symbol \\ Symbol &\to& P \, | \, Q \, | \, R \, \dots \\
Komplex &\to& \neg Satz \, | \, (Satz \wedge Satz) \, | \, (Satz \vee Satz) \, | \, (Satz \Rightarrow Satz) \, | \, (Satz \Leftrightarrow Satz)
\end{eqnarray*}

\subsubsection*{Semantik}
Wahrheitsgehalt im Modell $M$:
\begin{enumerate}
\item $True = wahr$, $False = falsch$ $\forall M$, wird für alle Symbole in $M$ spezifiziert
\item Wahrheitsgehalt aller anderen Sätze rekursiv:
\end{enumerate}

\subsection{Begriffe}

\begin{description}
	\item[Literal]
	\item[Disjunktion]
	\item[Konjunktion]
	\item[KNF] und-verknüpfte Disjunktionen
	\item[DNF] oder-verknüpfte Konjunktionen
\end{description}

\subsection{Deduktionsalgorithmen}

\subsubsection{Resolution}

\begin{description}
	\item[Modus Ponens] $$\frac{\alpha \Rightarrow \beta \, , \, \alpha}{\beta}$$
	\item[Und-Elimination] $$\frac{\alpha \wedge \beta}{\alpha} \qquad \frac{\alpha \wedge \beta}{\beta}$$
	\item[Einheits-Resolution] Seien $p_1,\dots,p_k$ und $q$ Literale mit $q = \neg p_i$. Dann gilt: $$\frac{p_1 \vee \dots \vee p_k \, , \, q}{p_1 \vee \dots \vee p_{i-1} \vee p_{i+1} \vee \dots \vee p_k}$$
	\item[Allg. Resolutionsregel] Seien $p_1,\dots,p_k$ und $q_1,\dots,q_n$ Literale mit $p_i = \neg q_j$. Dann gilt: $$\frac{p_1 \vee \dots \vee p_k \, , \, q_1 \vee \dots \vee q_n}{p_1 \vee \dots \vee p_{i-1} \vee p_{i+1} \vee \dots \vee p_k \vee q_1 \vee \dots \vee q_{j-1} \vee q_{j+1} \vee \dots \vee q_n}$$
\end{description}

Jeder vollständige Suchalgorithmus, kombiniert mit der Resolutionsregel,
\begin{itemize}
\item kann jede Schlussfolgerung ableiten, die aus jeder Wissensbasis der Aussagenlogik folgt.
\item bildet die Basis für Familien von vollständigen Deduktionsalgorithmen.
\item beweist $WB \models \alpha$ durch Widerlegung von $(WB \wedge \neg \alpha)$.
\end{itemize}
Voraussetzung: Sätze müssen in konjunktiver Form vorliegen.
\begin{description}
\item[Klausel\index{Klausel}:] Eine Disjunktion von Literalen $$(l_1 \vee \dots \vee l_n)$$, geschrieben als Menge.
\end{description}

\subsubsection{Resolutionsalgorithmus}

Zeige $WB \models \alpha$ durch Widerlegung von $(WB \wedge \neg \alpha)$
\begin{center}
\begin{tabular}{l}
$klauseln$ $:=$ Menge Klauseln in KF $(WB \wedge \neg \alpha)$ \\
$neu$ $:=$ $\{\}$ \\
\verb|loop:| \\
\verb|  for all | $C_m$, $C_n$ \verb| in | $klauseln$: \\
\verb|    |$res :=$ \verb| resolution(|$C_m$, $C_n$\verb|)| \\
\verb|    if | $\emptyset$ \verb| = | $res$: \verb| return | $true$ \\
\verb|    |$neu := neu \cup res$ \\
\verb|  if | $neu \subseteq klauseln$: \verb| return | $false$ \\
\verb|  |$klauseln := klauseln \cup neu$
\end{tabular}
\end{center}

Beispiel:
% Bild Folie 34 (Block)
\mypic{10}{bsp_resolution}

\subsection{Horn-Klausel}

\begin{description}
\item[Horn-Klausel\index{Horn-Klausel}:] Disjunktion von Literalen, von denen höchstens eins positiv ist: $$(\neg a \vee \neg b \vee c)$$ lasst sich schreiben als: $$\underbrace{(a \wedge b)}_{Koerper} \Rightarrow \underbrace{c}_{Kopf}$$
\end{description}
Spezialtypen von Horn-Klauseln:
\begin{itemize}
\item keine negativen Literale: Fakt, Axiom
\item genau ein positives Literal: Definition
\item kein positives Literal: Integritätseinschränkung
\end{itemize}
Eigenschaften von Horn-Klausel-basierten Wissensbasen:
\begin{itemize}
\item einfach visualisierbar: Und-Oder-Graph
\item Inferenz durch Vorwärts- und Rückwärtsverkettung, sehr natürliche und leicht verständliche Algorithmen
\item Folgerungsentscheidung kann in linearer Zeit geschehen!
\end{itemize}
\begin{description}
\item[Und-Oder-Graph\index{Und-Oder-Graph}:] einfache Visualisierung eines Horn-Klausel-Systems.
\end{description}

% Bild Folie 39
\mypic{7}{und_oder_graph}

Vorwärtsverkettung:
\begin{itemize}
\item gehe von Fakten aus
\item arbeite dich vorwärts durch den Baum
\item Ergebnis: alle durchführbaren Folgerungsentscheidungen
\end{itemize}
Rückwärtsverkettung:
\begin{itemize}
\item gehe von Anfrage aus
\item arbeite dich rückwärts durch den Baum
\item stoppe bei bekannten Fakten (Ground Truth, Axiome)
\item Ergebnis: Wahrheitsgehalt der Anfrage
\end{itemize}
\subsection{DPLL}
\begin{description}
\item[Davis-Putnam-Logemann-Loveland\index{Davis-Putnam-Logemann-Loveland} (DPLL):] Rekursiver Modellprüfungs-Algorithmus mit folgenden Verbesserungen:
\begin{itemize}
\item früher Abbruch: benutzt $(A \vee B) \wedge (A \vee C) = true$, sobald $A = true$
\item Einheitsklauseln (Klauseln mit nur einem Literal): Konjunktionen mit Einheitsklauseln sind nur wahr, wenn die Einheitsklauseln wahr sind \\ $\Rightarrow$ weitere Einschränkung des Suchraums
\item reine Symbole: in $(A \vee \neg B)$, $(\neg B \vee \neg C)$, $(C \vee A)$ sind $A$ und $\neg B$ rein und $C$ unrein \\ Suche nach reinen Symbolen schränkt Raum möglicher Modelle stark ein!
\end{itemize}
\end{description}
Algorithmus:
\begin{center}
\begin{tabular}{l}
\verb|funktion| $DPLL(k,s,m)$: \\
\verb|  # | $k$ Klauselmenge, $s$ Symbolmenge, $m$ Modell \\
\verb|  if| alle Klauseln wahr in $m$: \verb|return | $true$ \\
\verb|  if| eine Klausel falsch in $m$: \verb|return | $false$ \\
\verb|  |$P, \, wert :=$ \verb| FINDE_EINHEITSKLAUSEL(|$k,s,m$\verb|)| \\
\verb|  if | $P$ nicht leer: \\
\verb|    return | $DPLL(k,s-P,$\verb|SETZE_IN_MODELL(|$P,wert,m$\verb|)|$)$ \\
\verb|  |$P, \, wert :=$ \verb| FINDE_REINES_SYMBOL(|$k,m$\verb|)| \\
\verb|  if | $P$ nicht leer: \\
\verb|    return | $DPLL(k,s-P,$\verb|SETZE_IN_MODELL(|$P,wert,m$\verb|)|$)$ \\
\verb|  |$P :=$ \verb| ERSTES(|$s$\verb|); | $rest :=$ \verb| REST(|$s$\verb|)| \\
\verb|  return |$DPLL(k, rest,$\verb|SETZE_IN_MODELL(|$P,true,m$\verb|)|$)$ \verb| or| \\
\verb|         |$DPLL(k, rest,$\verb|SETZE_IN_MODELL(|$P,false,m$\verb|)|$)$
\end{tabular}
\end{center}
DPLL zur Überprüfung von Erfüllbarkeit:
\begin{center}
\begin{tabular}{l}
\verb|function| $DPLL_ERFUELLBAR(s):$ \\
\verb|  #| $s:$ Satz der Aussagenlogik \\
\verb|  | $klauseln :=$ Klauselmenge der KNF von $s$ \\
\verb|  | $symbole :=$ Menge der Symbole aus $s$ \\
\verb|  return| $DPLL(klauseln, symbole, \{\})$
\end{tabular}
\end{center}

\subsection{Prädikatenlogik}

Prädikate $P$, Funktionen $f$ etc möglich.

\subsubsection{Substitution}

\subsubsection{Resolution}

\section{Planung}

\subsubsection*{Repräsentation von Plänen}

Probleme formuliert, sodass
\begin{itemize}
\item (Existenz eines) Lösungsplan einfach
\end{itemize}
$\to$ Einschränkungsregeln für formelle Spezifikation von Zuständen, Zielen, Aktionen.

\subsection{STandford Research Institute Problem Solver (STRIPS)\index{STRIPS}}

(teilweise eingeschränkt)

\begin{description}
	\item[Zustandsrepräsentation] Konjunktion positiver aussagenlogischer Literale und Literale 1. Ordnung (L1) [fkt. und variablenfrei] $$Blau \wedge Rund \wedge IstTasse(T1)$$
	\item[Closed World Assumption] nicht im Zustand vorkommend $\to$ falsch
	\item[No Negative Feature] keine Negationen in Zustand / Vorbedingung (in Nachbedingung mgl) \\
	In Vorbed: $$b \quad \Rightarrow \quad \begin{array}{l} a \textrm{ darf nicht im Weltmodell vorkommen} \\ b \textrm{ muss im Weltmodell vorkommen} \end{array}$$
	In Nachbed: $$\neg c \wedge d \quad \Rightarrow \quad \begin{array}{l} \textrm{entferne } c \textrm{ aus dem Weltmodell} \\ \textrm{füge } d \textrm{ dem Weltmodell hinzu} \end{array}$$
	\item[Zielrepräsentation] teilweise spezifiziert, Konjunktion positiver Literale, erfüllt $\leftrightarrow$ Zustand enthält alle Literale des Ziels
	\item[Aktion] A = (Name $N_A$, Parameter $P_A$, Vorbed. $V_A$, Effekte $E_A$) \underline{oder} 
\begin{center}
\begin{tabular}{lll}
\textbf{Action} ( & \multicolumn{2}{l}{$StelleAuf(Obj1, Obj2),$} \\
& \textbf{Vorbed:} & $IstUntersatz(Obj2) \wedge$ \\ && $Auf(Obj1, Tisch) \wedge$ \\ && $Auf(Obj2, Tisch),$ \\
& \textbf{Effekt:} & $\neg Auf(Obj1, Tisch) \wedge$ \\ && $Auf(Obj, Obj2)$ \qquad \quad )
\end{tabular}
\end{center}
	\item[Anwendbarkeit] Aktion anwendbar $\equiv$ Modell $M$ ist hat Vorbedingung $V_A$
	\item[Ergebnis] $\textit{Ergebnis}(M, A) = M + \{ \textit{pos. Literale von } E_A) \} - \{ \textit{neg. Literale von } E_A \}$
\end{description}

Einschränkunen von STRIPS:
\begin{itemize}
\item Literale müssen funktionsfrei sein
\begin{itemize}
\item endliche Grammatik, Aktion $\equiv$ endliche Konjuktion, Lösbarkeitsbeweis einfach
\end{itemize}
\item \underline{nur} positive Literale , "{}geschlossene Welt"{}
\begin{itemize}
\item einfachere Planung aber Falsch-Sachverhalte nur implizit
\end{itemize}
\item keine Quantisierung
\end{itemize}
$\Rightarrow$ STRIPS-Aussagen oft lang und unübersichtlich

\subsection{Action Description Language (ADL)\index{ADL}}

\begin{description}
	\item[offene Welt] nicht im Zustand = unbekannt
	\item[Negative Features] negative Literale / Konjunktionen erlaubt
	\item[Ergebnis] Effekt $\neg P \wedge Q \rightarrow M' = M + \neg P + Q - P - \neg Q$
	\item[Gleichheit] "{} $=$ "{} nun dabei, $\textit{StelleAuf}(T1,T1)$ nicht mehr möglich
	\item[Typüberprüfung] In STRIPS explizit, in ADL implizit möglich:
\begin{center}
\begin{tabular}{lll}
\textbf{Action} ( & \multicolumn{2}{l}{$StelleAuf(Obj1 \, : \, Untersatz, \,\, Obj2 \, : \, Manipulierbar),$} \\
& \textbf{Vorbed:} & $Auf(Obj1, Tisch) \wedge$ \\ && $Auf(Obj2, Tisch),$ \\
& \textbf{Effekt:} & $\neg Auf(Obj1, Tisch) \wedge$ \\ && $Auf(Obj, Obj2)$ \qquad \quad )
\end{tabular}
\end{center}
\end{description}

\subsection{Heuristik}

\textbf{Suche im Zustandsraum}
\begin{itemize}
\item einfach, da Effekte bijektiv und keine Funktionssymbole \\ $\Rightarrow$ endlicher Zustandsraum \\ $\Rightarrow$ Standard-Algorithmen zur Baumsuche (z.B. A*)
\item \underline{aber} Zustandsraum oft sehr groß $2^n$, naiv = ineffizient $\to$ Heuristik / Segmentierung in Teilprobleme
\end{itemize}

\textbf{Heuristiken für Zustandsraumsuche}
\begin{itemize}
\item versuche, die Suchtiefe einzuschränken
\item Def. \textbf{Heuristikfunktion}\index{Heuristikfunktion}: $$h = H(M_0,M_1)$$ angenommene maximale Anzahl von Aktionen, um von $M_0$ zu $M_1$ zu kommen, optimistisch, darf über- aber nicht unterschätzen
\item beschränke Suche auf Teilbaum der Tiefe $h$
\item wenn nicht gefunden: suche je eine Ebene tiefer
\item Problem: finde möglichst gute Heuristikfunktionen, denn allg. für Planungsprobleme existiert nicht
\item Bsp: Suche im Zustandsraum $\to$ Anzahl noch zu erfüllender Literale als Heuristik?? $\to$ \underline{Nein}, denn wenn eine Aktion mehrere Literale erfüllt, überschätzt die Heuristik evtl den verbleibenden Aufwand
\end{itemize}

\subsubsection{Greedy}

Immer vom Zustand, der laut Heuristik am nächsten am Ziel ist.

\begin{itemize}
	\item häufig gut, aber große Umwege mgl, i.d.R. nicht optimal
\end{itemize}

\subsubsection{A* - Algorithmus (allg. für Graphen)}

Expandiere Knoten, für den Summe $$\textit{geschätzte Summe zum Ziel} + \textit{bisher zurückgelegte Strecke von Start}$$
minimal ist.
(berücksichtige \underline{alle} nicht expandierten Knoten)

\subsubsection{Partial Order Planning}

\begin{itemize}
	\item Aktionen nur soweit ordnen wie nötig (partial order) $\to$ Planung flexibler
	\item Planung nicht im Zustandsraum, sondern im Raum der gültigen PO-Pläne
	\begin{itemize}
		\item Aktionen, Ordnungsbed., Abhängigkeiten, offene Vorbed.
		\item Pseudoaktionen Start / Ziel
		\begin{itemize}
			\item Effekte Start = Literale des Startzustands
			\item Vorbed. Ziel = Literale des Zielzustands
		\end{itemize}
		\item zu Beginn Start < Ziel, keine Abhängigkeiten, offene Vorbed. als Ziel
	\end{itemize}
\end{itemize}

Für alle neuen Pläne:
\begin{algorithmic}
	\While{$\exists$ offene Vorbed., die nicht durch Startzustand erfüllt werden}
		\State wähle bel. offene Vorbed. $P$ einer Aktion $B$
		\ForAll{A | Effekte von A erfüllen P}
			\State ergänze Abhängigkeiten um $A \rightarrow_P B$
			\State ergänze Ordnungsbed. um $A < B$
			\If{$A \notin \textit{Plan}$}
				\State Füge $A$ zu Aktion hinzu
				\State Füge Ordnungsbed. $\textit{Start} < A$ und $A < \textit{Ziel}$ hinzu
			\EndIf
		\EndFor
	\EndWhile
\end{algorithmic}

Gültigkeit eines PO-Plans
\begin{itemize}
	\item Ordnungsbed. + Abh. berücksichtigt
	\item keine Zyklen $A < \dots < A$
	\item wenn Abh. $A \rightarrow_P B$ ex., dann $P$ nicht durch Aktion $C$ zwischen $A$ und $B$ negiert werden !
\end{itemize}

Beschleunigung
\begin{itemize}
	\item Literal, das wenigste Aktionen benötigt $\to$ geringer Verzweigungsgrad
	\item In der Praxis beachte zwei Spezialfälle : \\ Literal gar nicht erfüllbar $\to$ Abbruch \\ Literal nur durch eine Möglichkeit erfüllbar $\to$ Sofort
\end{itemize}

\subsubsection{Planungsgraph}

Polynomieller, spezieller Algorithmus.
\begin{itemize}
	\item Abwechselnde Schritte $L_0$, $A_0$, $L_1$, $\dots$ (Literale die wahr sein könnten (einmal unverändert, einmal Aktionsergebnisse), Aktionen die mit geg. Vorbed. erfüllt werden könnten)
	\item mutex: zwischen Literalen, zwischen Aktionen \\
	Literale: komplementär oder wenn sie nur aus sich gegenseitig ausschließenden Aktionen resultieren \\
	Aktionen: komplementäre Effekte, Effekt der einen zerstört Vorbed. der anderen, Vorbed. enthalten komplementäre Literale
\end{itemize}

\begin{algorithmic}
	\State Init: Alle Literale des Startzustands
	\While{Literale verändern sich nicht}
		\State alle mgl. Aktionen eintragen
		\State mutexe eintragen
	\EndWhile
\end{algorithmic}

$\rightarrow$ Erster Abschnitt, in denen Literale ohne Mutex vorkommen $\equiv$ optimistischer Schätzwert
$\rightarrow$ Im Planungsgraph nicht erreicht $=$ unerreichbar, aber Umkehrung gilt nicht

\subsection{Umweltmodell\index{Umweltmodell}}

$$\textit{Umweltmodell} : \textit{Reale Umwelt} \to \textit{innere Repräsentation}$$

\begin{center}
Semantik + Geometrie + Topologie
\end{center}

\begin{center}
2D , 2 $\frac{1}{2}$D (senkrechtes Anheben), 3D
\end{center}

\subsubsection{Umweltbedingungen}

Statisch vs. Dynamisch (mit Positionsänderungen) \\

Bekannte (alle Objekte vorhanden) vs. Unbekannte Umwelt (Kartographierung)

\subsubsection{Objektmodellierung}

Darstellung der Objekte der realen Umwelt mittels

\begin{description}
	\item[Kantenmodell] Ermittlung markanter Punkte, Verbindung durch geeignete Kanten
	\item[Oberflächenmodell] Nachbildung der Objektoberflächen (Polygone, mathematische Grundflächen, Bezier-Flächen, Freiformflächen [Patches])
	\item[Volumenmodell] Begrenzungsflächenmodell, Grundkörperdarstellung, Zellzerlegung (Oct-Tree)
	\item[Bsp. Constructive Solid Geometry] Aus parametrisierbaren Grundflächen, mittels boolescher Operatoren ($\to$ hoher algorithmischer Aufwand)
\end{description}

% Nicht in Vorlesungsfolien
%(Erweiterung des Ansatzes der Bezierkurven): \\
%gegeben ist ein Gitter von Führungspunkten $P_{ij}$ mit $0 \leq i \leq N$ und $0 \leq j \leq M$. Damit ist die Fläche beschrieben durch $$F(u,v) = \sum\limits_{i=0}^N \sum\limits_{j=0}^M P_{ij} \cdot B_{i,N}(u) \cdot B_{j,M}(v)$$ mit
%\begin{eqnarray*}
%B_{i,N}(u) &=& (1-u) B_{i,N-1}(u) + uB_{i+1,N-1}(u) \quad , \quad B_{i,0} = 1 \\
%B_{j,M}(v) &=& (1-v) B_{j,M-1}(v) + uB_{j+1,M-1}(v) \quad , \quad B_{j,0} = 1
%\end{eqnarray*}
%Die $B_{i,N}$ bzw. $B_{j,M}$ heißen auch Bernsteinpolynome.

%\begin{description}
%\item[Geometrisches Modell:] Anwendungen:
%\begin{itemize}
%\item Bahnplanung feinkörnig
%\item aktives Messen
%\item Objekterkennung (Basisobjekte)
%\item fahren von $x_s,y_s,z_s$ nach $x_e,y_e,z_e$
%\end{itemize}
%\item[Topologisches Modell:] Anwendungen:
%\begin{itemize}
%\item Planung (Manipulationsplanung, mittlere Körnigkeit, Bewegungsplanung)
%\item "{}fahre von Raum1 nach Raum5"{}
%\end{itemize}
%Über topologische Modelle:
%\begin{itemize}
%\item Anordnung von Objekten und Umwelt relativ zueinander gespeichert
%\item abgeleitet aus geometrischen Modellen: \\ Pfade z.B. aus Voronoi.Diagrammen, Quadtree, Potentialfeldern
%\item grobe Aktionsplanung aus topologischen Modellen
%\item kurzfristige Verwendung anderer Modelle bei unerwarteten Hindernissen (geometrischer Bodenplan, ad-hoc erzeugte Kantenmodelle aus Laserscan, Kamera)
%\end{itemize}
%\item[Semantisches Modell:] Anwendung: Planung auf Aufgabeneben $\to$ "{}fahre durch alle Büros"{} \\
%Über semantische Modelle:
%\begin{itemize}
%\item besondere Bedeutung im Rahmen der Mensch-Maschine-Kommunikation
%\item belegte Fläche als Objekt klassifiziert (z.B Schrank, Stuhl)
%\item eingeordnet in topologische Beschreibung
%\item Objekteigenschaften verwertbar: Objektzustand (Tür offen, Tasse leer); äußere Erscheinung des Objekts (Form, Farbe) als Diskriminator und Hilfe für die Sensorik; geometrisches Objektmodell für Greifplanung etc.
%\end{itemize}
%zugehörig zu semantischen Umweltbeschreibungen:
%\begin{itemize}
%\item Funktion eines Objektes
%\item Landmarken
%\item geometrische Objekte
%\item topologische Umweltdarstellungen
%\item Positionen
%\end{itemize}
%\end{description}
%
%Umweltrepräsentation der Information:
%\begin{itemize}
%\item Pfade \item Freiraum \item Objekte \item gemischte Modelle
%\end{itemize}

%\subsubsection*{Pfade}
%
%Normal: 2-dimensionaler Raum
%\begin{itemize}
%\item Polygonale Beschreibung der Objekte
%\item Pfad: lineare oder nichtlineare Verbindung zweier Punkte im Operationsraum
%\item Speicherung der Umwelt und Planungsinformation
%\item Repräsentation als ungerichteter Graph
%\end{itemize}
%Sichtbarkeitsgraphen:
%\begin{itemize}
%\item einfache Umweltdarstellung, partielle Modellierung
%\item kollisionsfreie Bewegung nur auf gespeicherten Pfaden
%\item Sichtbarkeitsgraphen zur automatischen Generierung von Pfaden
%\end{itemize}
%Erstellung eines Sichtbarkeitsgraphen:
%\begin{center}
%\begin{tabular}{l}
%\verb|for i = 1..#Obj| \\
%\verb|  for j = 1..#Objekte[i]| \\
%\verb|    for k = i..#Obj| \\
%\verb|      for l = 1..#Objekte[k]| \\
%\verb|        if (Sichtbarkeitstest(objekte[i][j], objekte[k],[l] == ok)| \\
%\verb|          neuer Pfad(i,j,k,l)|
%\end{tabular}
%\end{center}
%\begin{description}
%\item[Voronoi-Diagramme\index{Voronoi-Diagramme}:] Fahrwege liegen auf maximalem Abstand zu den Hindernissen. Das Voronoi-Diagramm ist der duale Graph der Delaunay-Triangulation.
%\end{description}

\subsubsection{Umgebungskarte}

Durch Sensordaten.

\begin{description}
	\item[Freiraum] $F_R$ Gesamter Raum - Hindernisse
	\item[Konfigurationsraum] $K_R$: Raum aller möglichen Konfigurationen vom Roboter (bspw. Anpassung zur Unterstützung der Planung, Hindernisse vergrößern)
	\item[Hindernisraum] $H$ Raum der Hindernisse
\end{description}

%\subsubsection*{Freiraum}
%
%\begin{itemize}
%\item Projektion der realen Welt in 2D/3D-Grundrissdarstellung
%\item kollisionsfrei befahrbare Freiräume in geeignete Bereiche zerlegt
%\item vorhandene Objekte und Hindernisse nicht berücksichtigt
%\end{itemize}
%Freiraumgraph:
%\begin{itemize}
%\item Knoten: Freiraumbereiche
%\item Kanten: Verbindung der Gebiete
%\end{itemize}
%Vorteile:
%\begin{itemize}
%\item geringere Komplexität der Wegplanung
%\item reduzierter Zeit- und Rechenaufwand
%\item Fahrsicherheitsüberlegungen deutlich vereinfacht
%\item Verbesserung der Anpassungsfähigkeit der Algorithmen an verschiedene Umwelten
%\end{itemize}
%Darstellungsformen der Freiflächen:
%\begin{itemize}
%\item Kacheln (Quader)
%\begin{itemize}
%\item Darstellung von besetzten Räumen durch orthogonale begrenzende Linien (Näherungen)
%\item Verlängerung der Linien bis zum nächsten besetzten Raum
%\item bildet Mosaik von besetzten und freien Kacheln/Quadern
%\item freie Kacheln $\to$ Graph, durch den Bewegung möglich ist
%\end{itemize}
%Quadtree-/Octtree-Aufteilung:
%\begin{itemize}
%\item hierarchische Aufteilung in freie und belegte Zellen
%\item Aufteilung teilbelegter Zellen in 4 (2D) bzw. 8 (3D) Unterzellen
%\item Resultat: Baumstruktur
%\item Pfadplanung: Aufstieg von Start- und Zielknoten bis zum ersten gemeinsamen Knoten; Liste besuchter Knoten: Pfad
%\end{itemize}
%\item konvexe Polygone (Polyeder)
%\end{itemize}

%\subsubsection*{Modelle}
%
%Aufgliederung des benötigten Wissens:
%\begin{itemize}
%\item Ausführung von Aufgaben
%\begin{itemize}
%\item semantisches Aufgabenmodell
%\item Umweltmodell vorher, nachher
%\end{itemize}
%\item Bewegung über Grund in 2D, Bewegung des Arms in 3D
%\begin{itemize}
%\item statische Umweltkarte (geometrisch und topologisch)
%\item dynamische Hinderniserfassung in 3D
%\item Freiraum-, Hindernismodell
%\end{itemize}
%\item Manipulation von Objekten
%\begin{itemize}
%\item Objektpositionen
%\item geometrische und topologische Objektmodelle
%\end{itemize}
%\end{itemize}

\subsection{Geometrisches Planen}

\subsubsection*{Bahnplanung\index{Bahnplanung}}

Geg.: Karte des Raumes \\
Ges.: Wege, potentiell der kürzeste \\

Ansatz: Wegenetz anlegen, als Graph repräsentieren, Wegsuche im Graph

\paragraph{Polygonzerlegung}

Schnell, aber evtl. große Umwege

\begin{enumerate}
	\item Von jeder Hindernissecke eine Trennlinie nach oben/unten bis zur nächsten Hinderniskante (komplett frei \underline{oder} komplett unpassierbar)
	\item Mittelpunkte verwenden um Graph zu bilden, A* Algorithmus
\end{enumerate}

\paragraph{Sichtgraph}

Sichtgraph erstellen, Kanten sind alle Kanten der Hindernisse \underline{und} Sichtlinien zu allen anderen Ecken. ($\to$ A* - Algorithmus) \\

Gitter als Annäherung der Hindernisse! \\

$\rightarrow$ kürzeste Wege in 2D (nicht 3D!), wenn Hindernisse als Polygone darstellbar
$\rightarrow$ Hindernisse vergrößern, um Anfahrt zu ermöglichen

\paragraph{Quadtree (Octtrees)}

Raum unterteilt in Quadrat. Quadrat kann frei, unpassierbar oder gemischt sein. Gemischte aufteilen. (bis Untergrenze mgl, dann gemischte $\equiv$ unpassierbar) \\

$\rightarrow$ so genau wie nötig!

\paragraph{Voronoi-Diagram}

Für Maximalabstand zu Hindernissen. Grenzlinie so, dass alle Punkte bei nähster Fläche. ($\to$ 2 punktförmige = Gerade, Punkte+Gerade = Parabel) \\
$\rightarrow$ definitiv hindernisfrei (max. Freiheit), große Umwege mgl.\\
$\to$ bei komplexen Hindernissen Potentialfeld

\paragraph{Potentialfeldmethode}

Potentialfeld im Konfigurationsraum definieren (1. globales Min. im Zielpunkt, linear ansteigend 2.Hindernisse sehr hohes Potential. Multiplizieren. Gradiend Descend) \\

$\rightarrow$ guter Kompromiss, aber lokale Minima

%
%Bewegung eines Roboters:
%\begin{itemize}
%\item Zustandsänderungen über der Zeit (Trajektorie)
%\item relativ zu stationärem Koordinatensystem (kartesischer Raum, Gelenkwinkelraum)
%\item häufig Gütekriterien, Neben-, Randbedingungen
%\end{itemize}
%Bekannt:
%\begin{itemize}
%\item $S_{start}$: Zustand zum Startzeitpunkt
%\item $S_{ziel}$: Zustand zum Zielzeitpunkt
%\end{itemize}
%Gesucht:
%\begin{itemize}
%\item $S_I$: Zwischenzustände (Stützpunkte)
%\item glatte, stetige Trajektorie
%\end{itemize}
%Bahnplanungsverfahren nach Zustandsraum:
%\begin{itemize}
%\item Gelenkwinkelzustandsraum (Konfigurationsraum)
%\item 3-dim. euklidischer Raum
%\item Sensorzustandsraum, Objektzustandsraum, \dots
%\end{itemize}
%Bahnplanungsverfahren nach Art des Roboters:
%\begin{itemize}
%\item Bahnplanung für Manipulatoren
%\item Bahnplanung für mobile Roboter
%\item Bahnplanung für Laufmaschinen und antropomorphe Systeme
%\item Greif- und Montageplanung
%\end{itemize}
%
%\subsubsection*{Bahnplanung im Gelenkwinkelraum}
%
%\begin{itemize}
%\item Trajektorie als Funktion der Gelenkwinkel
%\item Ausführung solcher Trajektoren durch
%\begin{itemize}
%\item Steuerung der Achsen unabhängig voneinander (Punkt zu Punkt) oder
%\item achsinterpolierte Steuerung (Bewegung aller Achsen beginnt und endet zum gleichen Zeitpunkt) erfolgen
%\end{itemize}
%\item Bahnverlauf muss im kartesischen Raum nicht definiert sein
%\item Vorteile: einfach; keine Singularitäten
%\end{itemize}
%
%\subsubsection*{Bahnplanung im kartesischen Raum}
%
%\begin{itemize}
%\item Trajektorie angegeben als Funktion der Endeffektorposition
%\item Funktionen z.B.: lineare Bahnen, Polynombahnen, Splines
%\item Vorteile:
%\begin{itemize}
%\item Verlauf der Trajektorie explizit in 3D
%\item einfach nachvollziehbar, visualisierbar
%\end{itemize}
%\item Nachteile:
%\begin{itemize}
%\item für jeden Punkt muss Gelenkwinkelrücktransformation berechnet werden
%\item Trajektorie nicht immer ausführbar (Arbeitsraumbegrenzung, Singularitäten des Roboters)
%\end{itemize}
%\end{itemize}
%
%\subsubsection*{Bahnplanungs-Schema}
%
%gegeben: Robotermodell (Geometrie, Kinematik), Umweltmodell
%
%\begin{enumerate}
%\item Berechnung des Konfigurationsraums $K$
%\item Berechnung des Hindernisraums $H$
%\item Berechnung des Freiraums $F = K \backslash H$
%\item Zerlegung des Freiraums in Unterräume
%\item Bahnplanung im Unterraum
%\item Integration der lokalen Lösungen in Gasamtlösung
%\end{enumerate}
%
%\subsubsection*{Konfiguration}
%
%\begin{description}
%\item[Konfiguration $k_R$:] beschreibt den Zustand eines Roboters $R$
%\begin{itemize}
%\item im euklidischen Raum durch Lage und Orientierung
%\item im Gelenkwinkelraum durch die Werte der Gelenke
%\end{itemize}
%\item[Konfigurationsraum $K_R$:] Raum aller möglichen Konfigurationen von $R$
%\item[Weg $w$ von $k_{start}$ bis $k_{ziel}$:] stetige Abbildung $$w \, : \, [0,1] \to K \quad \textrm{mit} \quad w(0) = k_{start} \, , \, w(1) = k_{ziel}$$
%\item[Arbeitsraum-Hindernis $h_{A,O}$:] Raum, der von einem Objekt $O$ im Arbeitsraum $A$ eingenommen wird
%\item[Konfigurationsraum-Hindernis $h_{K,O}$:] Raum, der von einem Objekt $O$ im Konfigurationsraum $K$ eingenommen wird
%\item[Hindernisraum $H$:] \quad $$H = \bigcup\limits_{O} h_{K,O}$$
%\end{description}
%
%\subsubsection*{Freiraum}
%
%\begin{itemize}
%\item Freiraum $F_R \, : \, F_R = K_R \backslash H$
%\item Aufwand für Freiraumberechnung: $O(m^n)$ mit
%\begin{itemize}
%\item $n$: Anzahl der Freiheitsgrade des Roboters
%\item $m$: Anzahl der Hindernisse
%\end{itemize}
%\item Deshalb oft approximative Verfahren zur Vereinfachung des Freiraums
%\begin{itemize}
%\item Sichtgraphen
%\item Quadtree, Octtree
%\end{itemize}
%\end{itemize}
%
%\subsection{Bahnplanung in 2D}
%
%Einfacher Algorithmus: "{}Strassenkarten"{} \\
%gegeben: 2-dim. Weltmodell, Start und Ziel \\
%gesucht: güstigste Verbindung von Start zu Ziel \\
%Lösung:
%\begin{enumerate}
%\item konstruiere Netz von Wegen $W$ in $F_R$
%\item bilde $k_{start}$ und $k_{ziel}$ auf $W$ ab: $(W(k_{start}), W(k_{ziel})$
%\item suche Weg $w$, der $W(k_{start})$ mit $W(k_{ziel})$ verbindet
%\end{enumerate}
%
%\subsubsection*{Wegkonstruktion}
%
%Wegkonstruktion mit
%\begin{itemize}
%\item Retraktionverfahren (z.B. Voronoi-Diagramm)
%\item Sichtgraphen
%\item Zellzerlegungsmethoden
%\end{itemize}
%Suche im Wegnetz mit z.B.
%\begin{itemize}
%\item A*-Algorithmus (Baumsuche)
%\item euklidischer Abstand
%\item Potentialfeld
%\end{itemize}
%
%\begin{description}
%\item[Retraktion\index{Retraktion}:] Sei $X$ eine Menge und $Y \subset X$. Eine surjektive Abbildung $$p \, : \, X \to Y$$ heisst Retraktion genau dann, wenn $p$ stetig ist und $p(y) = y$ für alle $y \in Y$ gilt. \\
%D.h. die Abbildung der Menge $X$ auf ihre Teilmenge $Y$, wobei die Menge $Y$ auf sich selbst abgebildet wird. Für die Bahnplanung gilt:
%\begin{itemize}
%\item $Y$ ist ein Netz von eindimensionalen Kurven (Wegenetz).
%\item Retraktionsmethoden unterscheiden sich in Wahl von $p$.
%\end{itemize}
%\end{description}
%
%\subsubsection*{Sichtgraphen}
%
%Konstruktion:
%\begin{itemize}
%\item Verbinde jedes Paar von Eckpunkten auf dem Rand von $F_R$ durch gerades Liniensegment, wenn das Segment kein Hindernis schneidet.
%\item Verbinde $k_{start}$ mit $k_{ziel}$ analog dazu.
%\end{itemize}
%Anmerkungen:
%\begin{enumerate}
%\item Wege sind nur "{}halbfrei"{} (nicht kollisionsfrei), da Hinderniskanten auch Wegsegmente sein können. \\ Abhilfe: Erweiterung der Hindernisse
%\item Wenn ein Weg gefunden ist, ist es auch der kürzeste Weg.
%\item Methode ist exakt, wenn Roboter nur 2 translatorische Freiheitsgrade hat und sowohl Roboter als auch Hindernisse durch Polygone dargestellt werden können.
%\item Methoden auch im $R^3$ anwendbar, jedoch sind die gefundenen Wege i.A. keine kürzesten Wege mehr.
%\end{enumerate}
%
%\subsubsection*{Kürzester Pfad im Graphen}
%
%vom Startknoten ausgehend:
%\begin{itemize}
%\item wähle Nachbarknoten $N_k$ so, dass Evaluationsfunktion $f(N_k)$ minimal
%\item suche von $N_k$ ausgehend weiter
%\item wenn $f(N_k)$ nicht mehr kleiner wird, mache weiter oben im Baum weiter
%\item Problem: $f(N_k)$ darf nicht vom Teilbaum an $N_k$ abhängen! \\ (Würden wir den kennen, bräuchten wir nicht suchen.)
%\item Lösung: verwende Heuristik $h(N_k)$
%\item beliebt für die Heuristikfunktion in 2D: euklidischer Abstand $$h(N_k) = ||N_k - N_{ziel}||$$
%\item rein heuristikbasierte Suche ist nicht optimal und führt nur unter Einschränkungen zum Ziel
%\item A*-Algorithmus\index{A*-Algorithmus}: \\ mit $g(N_k)$ Kosten zum Erreichen von $N_k$
%\item beweisbar optimal $$f(N_k) = g(N_k) + h(N_k)$$
%\end{itemize}










