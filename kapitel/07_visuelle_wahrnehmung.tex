% !TeX root = summary.tex

\chapter{Visuelle Wahrnehmung des Menschen}
Klausurrelevant Folien 1-24
\subsection{Bewegungserfassung I}
Problemstellung beim Human Motion Capture (HMC):
\begin{itemize}
	\item Eingabe: Sequenz von Bildern bzw. Bildpaaren oder Bildtupeln
	\item Ausgabe: Geschätzte Konfiguration (Gelenkwinkel) für jedes Frame bezüglich eines zuvor definierten Menschmodells
	\item Schwierigkeit: Hohe Dimensionalität des Suchraumes
\end{itemize}
\subsubsection{Menschmodell}
Menschmodell für HMC setzt sich zusammen aus
\begin{itemize}
	\item Kinematischem Modell
	\item Geometrischen Modell
	\begin{itemize}
		\item Meist aus Festköpern
		\item Optional: deformierbares Hautmodell
	\end{itemize}
\end{itemize}
Aus Gründen der Rechenzeit werden vereinfachte Modelle verwendet.
\subsubsection{Kinematisches Modell}
\begin{itemize}
	\item Definiert die Anzahl und Art der Gelenke
	\item Definiert die Segmentlängen zwischen der Gelenken
	\item Für die Erfassung wird die Schulter meist durch ein einzelnes Kugelgelenk modelliert
	\end{itemize}
\subsubsection{Geometrisches Modell}
\begin{itemize}
	\item Definiert die 3D-Form der einzelnen Segmente
	\item Übliche 3D-Primitive:
	
	\begin{itemize}
		\item Zylinder
		\item Kegelausschnitte (Kreis- oder Ellipsenförmiger Querschnitt)
	\end{itemize}
\end{itemize}
\subsubsection{Berechnung}
Berechnung der projizierten Kontur $\overline{P_1 P_2}$ und $\overline{P_3 P_4}$ eines Kegelausschnitts mit kreisförmigem Querschnitt.
Gegeben:
\begin{itemize}
	\item Porjektionszentrum=Ursprung $Z$
	\item Fußpunkt $c$
	\item Richtung $n$
	\item Länge $L$
	\item Radien $r$,$R$
\end{itemize}
Berechnung
\begin{itemize}
	\item $u=\frac{n\times c}{\left|n\times c\right|}$
	\item $c_t=c+L \cdot \frac{n}{\left|n\right|}$
	\item $p_{1,3}=c \pm R \cdot u$
	\item $p_{2,4}=c_t \pm r \cdot u$
\end{itemize}
\subsubsection{Bildbasiert mit Partikelfilter}
Den Kern bildet eine Wahrscheinlichkeitsfunktion, welche bewertet, wie gut eine gegeben Konfiguration des Menschmodells zu den aktuellen Beobachtungen (Bilddaten) passt.

Hinweise (engl. Cues) für die Bewertung, die aus den Bilddaten gewonnen werden können sind:
\begin{itemize}
	\item Region Cue \cite{Deutscher.2000}
	\item Kanten Cue \cite{Deutscher.2000}
	\item Distanz Cue [Azad]
\end{itemize}
\subsubsection{Region Cue}
\begin{itemize}
	\item Benötigt Segmentierung der Person vom Hintergrund
	\item Bewertet den Abgleich des Segmentierungsergebnisses mit der Projektion der Körpersegmente
	\item Hierzu werden Punkte innerhalb der projizierten Kontur überprüft
	\item Bewertungsfunktion:
\end{itemize}
\subsection{Iterative Closet Point (ICP)}
\subsection{Bewegungserfassung II}












