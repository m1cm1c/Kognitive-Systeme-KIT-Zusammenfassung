% !TeX root = summary.tex

\chapter{Visuelle Wahrnehmung des Menschen}

\subsection{Human Motion Capture (HMC)}
Eingabe(Bilder) $\rightarrow$ Ausgabe(Geschätze Konfiguration/Gelenkwinkel bzgl. vordef. Menschmodell) \\
\textit{Schwierigkeit}: Hohe Dimensionalität des Suchraumes, Eingeschränkte Beobachtbarkeit

\subsubsection{Menschmodell}
\begin{itemize}
	\item Kinematischem und Geometrisches(fest, optional verformbar) Modell
	\item Vereinfacht aufgrund von Rechenzeit
\end{itemize}

\subsubsection{Kinematisches Modell}
\begin{itemize}
	\item Anzahl und Art der Gelenke, Segmentlänge dazwischen
	\item Für Erfassung: Schulter = 1 Kugelgelenk
\end{itemize}

\subsubsection{Geometrisches Modell}
\begin{itemize}
	\item 3D-Form der einzelnen Segmente
	\item Übliche in Zylindern, Kegelausschnitte (Kreis- oder Ellipsenförmiger Querschnitt)
\end{itemize}

\textbf{Sequential vs. parallel coordinated motion}

\subsubsection{Berechnung projezierte Kontur}
Projizierten Kontur $\overline{P_1 P_2}$ und $\overline{P_3 P_4}$ eines Kegelausschnitts mit kreisförmigem Querschnitt.
Gegeben: Porjektionszentrum = Ursprung $Z$, Fußpunkt $c$, Richtung $n$, Länge $L$, Radien $r$,$R$

\begin{itemize}
	\item $u=\frac{n\times c}{\left|n\times c\right|}$
	\item $c_t=c+L \cdot \frac{n}{\left|n\right|}$
	\item $p_{1,3}=c \pm R \cdot u$
	\item $p_{2,4}=c_t \pm r \cdot u$
\end{itemize}

\subsubsection{Bildbasiert mit Partikelfilter}
Wahrscheinlichkeitsfunktion bewertet, wie gut eine gegeben Konfiguration des Menschmodells zu den aktuellen Beobachtungen (Bilddaten) passt.

Hinweise (engl. Cues) für die Bewertung, die aus den Bilddaten gewonnen werden können sind:
\begin{itemize}
	\item Region Cue, Kanten Cue, Distanz Cue
\end{itemize}

\subsubsection{Region Cue}
\begin{itemize}
	\item Benötigt Segmentierung der Person vom Hintergrund
	\item Bewertet den Abgleich des Segmentierungsergebnisses mit der Projektion der Körpersegmente
	\item Hierzu werden Punkte innerhalb der projizierten Kontur überprüft
	\item Kanten Cue ( nur die Kanten, nicht alles )
	\item Problem: Nicht statische Kameras, Echtzeitanforderungen
\end{itemize}

\subsubsection{Distanc Cue}
\begin{itemize}
	\item mittels 3D Hand/Kopftracking
	\item Bewertung Distanz  erfasste Position des Tracking und der zu bewertende Konfiguration
	\item Bewertungsfunktion $w_d(I_d, P) = \sum_{i=1}^{|P|} |p_i - p_i'(I_d)|^2$
\end{itemize}

\subsection{Iterative Closet Point (ICP)}
Dichte 3D Punktwolken durch Volumenschnittverfahren

\begin{itemize}
	\item Hierarchische Registrierung jedes einzelnen Segments
	\item Berücksichtigung Gelenkwinkelbeziehungen ($\to$ sehr rechenaufwendig)
\end{itemize}

\subsection{Gesichtserkennung}
Detektion: Dort ist ein Gesicht, nicht welches.

\begin{itemize}
	\item unterschiedliche Objektklassen vs. versch. Objekte in einer Klasse
	\item Problem: Variation, Sicht Beleuchtung etc.
\end{itemize}

Ansätze:
\begin{itemize}
	\item Merkmalsbasiert (Merkmalvektor, markante Punkte, Abstände)
	\begin{itemize}
		\item Breite, Höhe der Lippen, Augenbrauen, Kinnform und Position...
	\end{itemize}
	\item Ansichtsbasiert (Holistisch, markante Regionen, statistisch)
\end{itemize}

Klassifikation mittels Mahalanobis-Distanz:
$$\Delta_j(x) = (x - m_j)^T \Sigma^{-1}(x - m_j)$$
wobei $x$ Anfragebild, $m$ Durchschnittsvektor von Person $j$, $\Sigma$ Kovarianzmatrix











